%
% Sample SBC book chapter
%
% This is a public-domain file.
%
% Charset: ISO8859-1 (latin-1) áéíóúç
%
\documentclass{SBCbookchapter}
\usepackage[brazil]{babel}
\usepackage[utf8]{inputenc}
\usepackage[T1]{fontenc}
\usepackage{tikz}
\usetikzlibrary{mindmap, shadows, trees}
\usepackage{float}
\setlength{\intextsep}{8pt}   % Espaço acima/abaixo de figuras no meio do texto
\setlength{\textfloatsep}{10pt} % Espaço entre texto e figuras flutuantes

\author{Alexandre Zucki Baciuk, João Pedro Gava Ribeiro e Rodrigo Belniok Czelusniak}
\title{Lightweight Cryptography}
\begin{document}

\maketitle

\begin{abstract}
This meta-paper describes the style to be used in articles and short
papers for SBC conferences. For papers in English, you should add just
an abstract and for the papers in Portuguese, we also ask for an
abstract in Portuguese (``resumo''). In both cases, abstracts should not
have more than 10~lines and must be in the first page of the paper.
\end{abstract}

\begin{resumo}
\begin{otherlanguage}{brazil}
Este meta-artigo descreve o estilo a ser usado na confecção de artigos
e resumos de artigos para publicação nos anais das conferências
organizadas pela SBC. É solicitada a escrita de resumo e abstract apenas
para os artigos escritos em português. Artigos em inglês, deverão
possuir apenas abstract. Nos dois casos, o autor deve tomar cuidado para
que o resumo (e o abstract) não ultrapassem 10~linhas cada, sendo que
ambos devem estar na primeira página do artigo.
\end{otherlanguage}
\end{resumo}

\section{Introdução}

A Internet das Coisas (ou \textit{Internet of Things} -- IoT, do inglês) é caracterizada por vários dispositivos interconectados que trocam informações entre si \cite{dutta:2019}. Esta consolidou-se como uma área de pesquisa robusta e abundante, a partir do avanço de suas aplicações em múltiplos domínios \cite{thakor:2020}, alguns dos quais foram apresentados na Figura \ref{fig:1}.

Então, para exemplificar, na saúde conectada os sensores de IoT possibilitam um monitoramento contínuo de pacientes e a rápida reação perante anormalidades. Já na casa inteligente, diversos eletrodomésticos, como a iluminação inteligente, simplificam o cotidiano de seus detentores \cite{sembroiz:2018}.

Assim, por mais que o termo tenha sido cunhado em 1999, empresas como a Cisco assumem que o conceito de IoT começou a fazer sentido em 2009, quando a quantidade de dispositivos conectados (10 bilhões) ultrapassou a população global. De certo modo, esta popularização se deve a melhorias aferidas em termos de eficiência, custo de produção e redução de tamanho de sensores \cite{sembroiz:2018}.

\begin{figure}[H]
    \centering
    \caption{Domínios de Aplicação da IoT}
    \label{fig:1}
    \begin{tikzpicture}[mindmap, grow cyclic, every node/.style=concept, concept color=blue!40, 
	level 1/.append style={level distance=3.5cm,sibling angle=45}]
    \node[concept color=blue!40] {Aplicações de IoT}
    child [concept color=red!60] {node {Cidade Inteligente}}
    child [concept color=orange!70] {node {Casa Inteligente}}
    child [concept color=yellow!60!orange] {node {Agricultura Inteligente}}
    child [concept color=green!50] {node {Rede Inteligente}}
    child [concept color=cyan!60] {node {Carro Conectado}}
    child [concept color=violet!60] {node {Saúde Conectada}}
    child [concept color=magenta!60] {node {Dispositivo Vestível}}
    child [concept color=brown!60] {node {Indústria 4.0}};
\end{tikzpicture}
\end{figure}

Logo, estima-se que em 2024 existiam 18 bilhões de dispositivos conectados na infraestrutura de IoT e projeta-se que existirão 40 bilhões até 2030 \cite{iotanalytics:2024}. Nesse sentido, um incremento crescente no número destes dispositivos impõe preocupações atinentes à segurança \cite{dutta:2019}, as quais podem ser expressas como segue:

\begin{itemize}
    \item Confidencialidade: apenas usuários autorizados podem acessar a informação.
    \item Disponibilidade: quando necessário, o dispositivo deve conseguir acessar a informação requerida.
    \item Integridade: deve-se assegurar que os dados são precisos.
    \item Autenticação: os elementos da rede possuem níveis de acesso diferente. Este é um aspecto complexo de ser implementado em IoT.
    \item Heterogeneidade: como os elementos da rede se distinguem em complexidade e fabricante, necessita-se de uma rede heterogênea.
\end{itemize}

Isto posto, surge a motivação para o desenvolvimento de abordagens de Criptografia Leve (ou \textit{Lightweight Cryptography} -- LWC), objeto de pesquisa do presente artigo, ou seja, os esquemas criptográficos que dependem de uma complexidade computacional e consumo de memória reduzidos. Sendo assim, a LWC é adequada para sistemas restritos em memória e poder computacional, os quais seriam incapazes de processar em tempo hábil os algoritmos criptográficos convencionais \cite{aissaoui:2023}.

\section{First Page}
The first page must display the paper title, the name and address of
the authors, the abstract in English and ``resumo'' in Portuguese (for
papers written in Portuguese). The title must be justified at the
left, in 20~point boldface font. Author names must be justified in the
same way, as shown in this example.

\section{CD-ROMs and Printed Proceedings}
In some conferences, the papers are published on CD-ROM while only the
abstract is published in the Proceedings. In this case, authors are
invited to prepare two final versions of the paper. One, complete, to
be published on the CD and the other, containing only the first page,
with abstract and ``resumo'' (for papers in Portuguese).

\section{Sections and Paragraphs}
Section titles must be in boldface, 13pt, flush left. There should be
an extra 12~pt of space before each title. The first paragraph of each
section should not be indented; the first lines of subsequent
paragraphs should be indented by 1.27~cm.

\subsection{Subsections}
The subsection titles must be in boldface, 12pt, flush left.

\section{Figures and Captions}
\label{sec:captionmargins}
Figures and tables captions should be centered if less than one line
(Figure~\ref{figone}), otherwise justified and indented by 0.8cm on
both margins, as shown in Figure~\ref{figtwo}. The font must be
Helvetica, 10~point, boldface, with 6~points of space before and after
each caption.

In tables, do not use colored or shaded backgrounds, and avoid thick,
doubled, or unnecessary framing lines. When reporting empirical data,
do not use more decimal digits than warranted by their precision and
reproducibility. Table caption must be placed before the table (see
Table~\ref{tabone}) and the font used must also be Helvetica,
10~point, boldface, with 6~points of space before and after each
caption.

Figure and table references must be composed by the chapter number and
a sequence number beginning in one (see the examples of
Figure~\ref{figone}, Figure~\ref{figtwo} and Table~\ref{tabone}).

\section{Images}
All images and illustrations should be in black-and-white, or gray
tones. The image resolution on paper should be about 600~dpi for
black-and-white images, and 150-200~dpi for grayscale images.  Do not
include images with excessive resolution, as they may take hours to
print, without any visible difference in the result.

\begin{table}[h!]
	\caption{Variables to be considered on the evaluation of
		interaction techniques.}
	\label{tabone}
	\begin{footnotesize}
	\begin{tabular}{|p{40mm}|p{55mm}|p{42mm}|}
	\hline
	\hspace*{\fill}\textbf{Tarefa}\hspace*{\fill} &
	\hspace*{\fill}\textbf{Variável}\hspace*{\fill} & 
	\hspace*{\fill}\textbf{Métrica utilizada}\hspace*{\fill}\\
	\hline
	Seleção &
	Distância do alvo &
	Virtual cubits \\ \cline{2-3}
	& Direção horizontal e vertical do alvo &
	Graus do arco \\ \cline{2-3}
	& Distância do objeto oculto &
	Virtual cubits \\ \cline{2-3}
	& Direção da oclusão &
	Esquerda/direita/cima/baixo \\
	\hline
	Posicionamento &
	Distância inicial &
	Virtual cubits \\ \cline{2-3}
	& Direções iniciais horizontal e vertical &
	Graus do arco \\ \cline{2-3}
	& Distância final &
	Virtual cubits \\ \cline{2-3}
	& Direções finais horizontal e vertical &
	Graus do arco \\ \cline{2-3}
	& Precisão vertical &
	Porcentagem de sobreposição \\ \cline{2-3}
	& Precisão horizontal &
	Porcentagem de sobreposição \\
	\hline
	Orientação &
	Distância &
	Virtual cubits \\ \cline{2-3}
	& Direções horizontal e vertical &
	Graus do arco \\ \cline{2-3}
	& Orientação inicial (3 angulos) &
	Graus do arco \\ \cline{2-3}
	& Orientação final (3 angulos) &
	Graus do arco \\ \cline{2-3}
	& Exatidão/precisão &
	Graus do arco \\
	\hline
	\end{tabular}
	\end{footnotesize}
\end{table}

\bibliographystyle{sbc}
\bibliography{sbc-template}

\end{document}
